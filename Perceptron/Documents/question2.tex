\section{Feature transformations}\label{sec:q2}

[10 points] Consider the concept class $C$ consisting of functions $f_r$ defined by a radius $r$ as follows:

\begin{equation*}
            f_r(x_1,x_2) = \begin{cases}
                +1 \quad 17x_1^4-16x_2^3\leq r \\
                -1 \quad \text{otherwise}
            \end{cases}
            \label{eq-0}
        \end{equation*}

Note that the hypothesis class is \textit{not} linearly separable in $\mathbb{R}^2$.\\

Construct a function $\phi(x_1,x_2)$ that maps examples to a new space, such that the positive and negative examples are linearly separable in that space. The answer to this question should consist of two parts: 
\begin{enumerate}
\item A function $\phi$ that maps examples to a new space.
\item A proof that in the new space, the positive and negative points are linearly separated. You can show this by producing such a hyperplane in the new space (i.e. find a weight vector \textbf{w} and a bias $b$ such that $\textbf{w}^T\phi(x_1,x_2)\geq b$ if, and only if, $f_r(x_1,x_2)=+1$.
\end{enumerate}
Hint: The feature transformation $\phi$ should not depend on $r$.

%%% Local Variables:
%%% mode: latex
%%% TeX-master: "hw2"
%%% End:
